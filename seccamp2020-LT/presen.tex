
\documentclass[dvipdfmx, dvipsnames, 11pt]{beamer}
% \documentclass[dvipdfmx, 12pt, aspectratio=169]{beamer} % 16:9
\usepackage{pgfpages}
\usepackage{bxdpx-beamer}


%%%% PDFのしおりの文字化け対策 %%%%
\usepackage{atbegshi}
\ifnum 42146=\euc"A4A2
\AtBeginShipoutFirst{\special{pdf:tounicode EUC-UCS2}}
\else
\AtBeginShipoutFirst{\special{pdf:tounicode 90ms-RKSJ-UCS2}}
\fi
%%%%

%%%% スライドの見た目 %%%%
% http://deic.uab.es/~iblanes/beamer_gallery/
\usetheme{Singapore} % Theme
\usecolortheme{default} % Color
\usefonttheme{professionalfonts} % Font
%%%%

%%%% フォントの設定 %%%%
\usepackage[utf8]{inputenc} % UTF-8
\usepackage[T1]{fontenc} % 8bit font
\usepackage{lmodern} % Latin Modern font
\usepackage{otf} % otf (font)
\usepackage{bm} % bold math
\renewcommand{\kanjifamilydefault}{\gtdefault} % fontfamily: Gothic
\renewcommand{\familydefault}{\sfdefault} % 英語フォント
\usepackage{txfonts}
%%%%

% %% ----- math -----
\usepackage{amssymb}
% %% ----- programing -----
% \usepackage{verbatim}
\newcommand{\code}[1]{ \texttt{\detokenize{#1}} }
\usepackage{listings,jlisting}
\newcommand{\HighLight}{\makebox[0pt][l]{\color[rgb]{1, 0.88, 0.66}\rule[-2pt]{1.0\linewidth}{8pt}}}
\newcommand{\HighLightI}{\only<1>{\HighLight}}
\newcommand{\HighLightII}{\only<2>{\HighLight}}
\newcommand{\HighLightIII}{\only<3>{\HighLight}}
\newcommand{\HighLightIV}{\only<4>{\HighLight}}
\newcommand{\HighLightV}{\only<5>{\HighLight}}
\lstset{language=C,
    basicstyle=\ttfamily\scriptsize,
    keywordstyle=\bfseries,
    showstringspaces=false,
    morekeywords={free, query, process, in, out, fun, reduc, forall, let},
    escapechar=\%,
}
% %% ----- tikz -----
\usepackage{tikz}
\usetikzlibrary{calc,positioning,arrows,fit}
\usetikzlibrary{arrows.meta}

\newcommand{\enc}{\mathrm{enc}}
\newcommand{\dec}{\mathrm{dec}}

\title{~\\~\\~\\~\\プロトコルの形式的安全性検証ツール\\ProVerif}
\subtitle{}
\author{~\\~\\~\\~\\@tex2e}
\date{}

\begin{document}

% ==============================================================================
\begin{frame}[plain]
  \frametitle{}
  \titlepage %表紙
\end{frame}

% ==============================================================================
\section{プロトコルの形式的安全性検証}

% --------------------------------------------------------------------
\begin{frame}[fragile]
  \frametitle{ProVerif概要}
  \vspace{1em}

  \begin{itemize}
    \item できること:
      \\~~~~ プロトコルをモデル化したコードを記述 
      \\~~~~ $\rightarrow$ ProVerifツールで実行
      \\~~~~ $\rightarrow$ 脆弱性あり・なしの判定
    \vspace{1em}
    \item わかること:
      \\~~~~ プロトコルのセキュリティ特性
      \\~~~~~~~~ 秘匿性、{\color{gray}真正性}、オフライン攻撃、前方秘匿性
    \vspace{1em}
    \item 今日のお話:
      \\~~~~ サンプルプロトコルで検証
      \begin{itemize}
        \item プロトコル$\alpha$
        \item プロトコル$\beta$
      \end{itemize}
  \end{itemize}
\end{frame}

% --------------------------------------------------------------------
\begin{frame}[fragile]
  \frametitle{ProVerif}
  \vspace{1em}

  \begin{itemize}
    \item プロトコルが安全かどうかを自動で証明
    \item 誰でも使えて無料\footnote{\tiny\url{http://proverif.inria.fr/}}
    \vspace{0.5em}
    \item spi計算の書き方に(少し)似ている
    \item 言語としてはOCamlに(少し)似ている
  \end{itemize}

  \vspace{1em}

  \begin{lstlisting}
    free c: channel.
    free message: bitstring [private].

    query attacker(message).

    process
      out(c, message);
      0
  \end{lstlisting}
\end{frame}

% ==============================================================================
\section{プロトコル\scriptsize$\alpha$}

% --------------------------------------------------------------------
\begin{frame}
  \frametitle{プロトコル{\LARGE$\alpha$}}

  \begin{center}
    \begin{tikzpicture}[node distance=2cm,auto,>=stealth']
      \node[draw=black] (server) {Server};
      \node[draw=black, left of=server, node distance=5cm] (client) {Client};
      \node[below of=server, node distance=4cm] (server_ground) {};
      \node[below of=client, node distance=4cm] (client_ground) {};
      %
      \draw (client) -- (client_ground);
      \draw (server) -- (server_ground);
      \node at ($(client)!0.2!(client_ground)$) [fill=white] {パスワード$p$, 平文$m$};
      \node at ($(server)!0.2!(server_ground)$) [fill=white] {パスワード$p$};
      \draw[->] ($(client)!0.55!(client_ground)$) -- node[above,scale=1,midway]{暗号化$\enc(m, p)$} ($(server)!0.60!(server_ground)$);
      \node at ($(server)!0.85!(server_ground)$) [fill=white] {復号};
    \end{tikzpicture}
  \end{center}

  \begin{center}
    このプロトコルが安全かどうかを\bf{ProVerif}で検証します
  \end{center}
\end{frame}

% --------------------------------------------------------------------
\begin{frame}[fragile]
  \frametitle{プロトコルのモデル化}
  \framesubtitle{ハッシュ関数}

  特徴:
  \begin{itemize}
    \item 一方向関数 : $f(m) \rightarrow h$
    \item 完全な暗号モデル\hspace{0em}\footnote{\tiny Dolve-Yaoモデル}\hspace{0em}において逆関数は存在しない
  \end{itemize}

  \begin{center}
    \begin{tikzpicture}[color=white!30!blue]
      \draw[-{Triangle[width=35pt,length=15pt]}, line width=20pt](0,0) -- (0,-1)
        node [midway, right=1em] {ProVerifのコード};
    \end{tikzpicture}
  \end{center}
  \vspace{-1em}

  \begin{lstlisting}[basicstyle=\ttfamily\footnotesize]
    fun hash(bitstring): bitstring.
  \end{lstlisting}
\end{frame}

% --------------------------------------------------------------------
\begin{frame}[fragile]
  \frametitle{プロトコルのモデル化}
  \framesubtitle{共通鍵暗号}

  特徴:
  \begin{itemize}
    \item 暗号化 : $\enc(m, k)$
    \item 復号  : $\dec(c, k)$
    \item 暗号化と復号で元に戻る : $\dec(\enc(m, k), k) = m$
  \end{itemize}

  \begin{center}
    \begin{tikzpicture}[color=white!30!blue]
      \draw[-{Triangle[width=35pt,length=15pt]}, line width=20pt](0,0) -- (0,-1)
        node [midway, right=1em] {ProVerifのコード};
    \end{tikzpicture}
  \end{center}
  \vspace{-1em}

  \begin{lstlisting}
    fun enc(bitstring, key): bitstring.
    reduc forall m: bitstring, k: key;
      dec(enc(m,k),k) = m.
  \end{lstlisting}
\end{frame}

% --------------------------------------------------------------------
\begin{frame}[fragile]
  \frametitle{プロトコル{\LARGE$\alpha$}のモデル化}
  \framesubtitle{クライアント--サーバ間の通信}

  \begin{lstlisting}
    (* クライアントA *)
    let clientA() =
      event beginA(msg);
      out(c, enc(msg, password));
      0.

    (* サーバB *)
    let serverB() =
      in(c, x: bitstring);
      let recvmsg = dec(x, password) in
      event endB(recvmsg);
      0.

    process
      ( (!clientA()) | (!serverB()) )
  \end{lstlisting}
\end{frame}

% --------------------------------------------------------------------
\begin{frame}[fragile]
  \frametitle{検証方法}
  \framesubtitle{秘匿性}

  限られた人しか情報にアクセスできないこと

  \vspace{2em}
  \begin{itemize}
    \item \code{query} : 検証クエリ
    \item \code{attacker(v)} : 攻撃者は変数$v$に到達可能か
  \end{itemize}

  \vspace{2em}
  \begin{lstlisting}
    (* 秘匿性の検証 *)
    query attacker(msg).
    query attacker(password).
  \end{lstlisting}
\end{frame}

% --------------------------------------------------------------------
\begin{frame}[fragile]
  \frametitle{検証結果}
  \framesubtitle{秘匿性}

  \begin{lstlisting}[basicstyle=\ttfamily\tiny, frame=single]
$ proverif -color protocol1.pv
(
    {1}!
    {2}out(c, enc(msg,password))
) | (
    {3}!
    {4}in(c, x: bitstring);
    {5}let recvmsg: bitstring = dec(x,password) in
    0
)
...
--------------------------------------------------------------
Verification summary:

Query not attacker(msg[]) is %\color{OliveGreen}true%.

Query not attacker(password[]) is %\color{OliveGreen}true%.
  \end{lstlisting}
  
  \begin{table}
    \centering
    \begin{tabular}{lll}
      秘匿性 & $\rightarrow$ あり & \color{OliveGreen}$\checkmark$ \\
      \color{gray}オフライン攻撃 &  &  \\
      \color{gray}前方秘匿性 &  &
    \end{tabular}
  \end{table}
\end{frame}

% --------------------------------------------------------------------
\begin{frame}[fragile]
  \frametitle{検証方法}
  \framesubtitle{オフライン攻撃}

  盗聴した内容をオフラインで解読すること

  \vspace{2em}
  \begin{itemize}
    \item \code{weaksecret v.}
      \\~~~~ 秘密値$v$のエントロピーが低いとき
      \footnote{\tiny 人間が覚えられる程度の文字列しかないとき(パスワードなど)}、
      \\~~~~ 攻撃者は変数$v$に到達可能か
  \end{itemize}

  \vspace{2em}
  \begin{lstlisting}
    (* オフライン攻撃の検証 *)
    weaksecret password.
  \end{lstlisting}
\end{frame}

% --------------------------------------------------------------------
\begin{frame}[fragile]
  \frametitle{検証結果}
  \framesubtitle{オフライン攻撃}

  \begin{lstlisting}[basicstyle=\ttfamily\tiny, frame=single]
$ proverif -color protocol1a.pv
...
The attacker tests whether dec(~M,@weaksecretcst) is fail knowing
~M = enc(msg,password).
This allows the attacker to know whether @weaksecretcst = password.
A trace has been found.
RESULT Weak secret password is false.
...
--------------------------------------------------------------
Verification summary:

%\HighLightII%Weak secret password is %\color{red}false%.%\footnote{\tiny 弱い秘密を使ったときプロトコルの安全性はない、という意味}%

Query not attacker(msg[]) is %\color{OliveGreen}true%.

Query not attacker(password[]) is %\color{OliveGreen}true%.
  \end{lstlisting}

  \begin{table}
    \centering
    \begin{tabular}{lll}
      \color{gray}秘匿性 & \color{gray}$\rightarrow$ あり & \color{OliveGreen}$\checkmark$ \\
      オフライン攻撃 & $\rightarrow$ 可能 & \color{red}$\times$ \\
      \color{gray}前方秘匿性 &  &
    \end{tabular}
  \end{table}
\end{frame}

% --------------------------------------------------------------------
\begin{frame}[fragile]
  \frametitle{検証結果}
  \framesubtitle{オフライン攻撃}

  \includegraphics[width=10cm]{graph/protocol1_weaksecret.pdf}
\end{frame}

% --------------------------------------------------------------------
\begin{frame}[fragile]
  \frametitle{検証方法}
  \framesubtitle{前方秘匿性}

  秘密鍵が漏洩しても、過去の暗号化通信が復号できないこと

  \vspace{2em}
  \begin{itemize}
    \item \code{phase 1; out(c, password)}
      \vspace{0.5em}
      \\~~~~ Phase 0 : パスワードで平文$m$を暗号化して送信
      \\~~~~ Phase 1 : パスワードを漏洩させる
      \\~~~~~~ パスワード漏洩後に攻撃者は平文$m$に到達可能か
  \end{itemize}

  \vspace{2em}
  \begin{lstlisting}
    (* 前方秘匿性の検証 *)
    process
      ( (!clientA()) | (!serverB()) | phase 1; out(c, password) )
  \end{lstlisting}
\end{frame}

% --------------------------------------------------------------------
\begin{frame}[fragile]
  \frametitle{検証結果}
  \framesubtitle{前方秘匿性}

  \begin{lstlisting}[basicstyle=\ttfamily\tiny, frame=single]
$ proverif -color protocol1b.pv
...
(
    {1}!
    {2}out(c, enc(msg,password))
) | (
    {3}!
    {4}in(c, x: bitstring);
    {5}let recvmsg: bitstring = dec(x,password) in
    0
) | (
    {6}phase 1;
    {7}out(c, password)
)
...
--------------------------------------------------------------
Verification summary:

Query not attacker_p1(msg[]) is %\color{red}false%.

Query not attacker_p1(password[]) is %\color{red}false%.
  \end{lstlisting}

  \begin{table}
    \centering
    \begin{tabular}{lll}
      \color{gray}秘匿性 & \color{gray}$\rightarrow$ あり & \color{OliveGreen}$\checkmark$ \\
      \color{gray}オフライン攻撃 & \color{gray}$\rightarrow$ 可能 & \color{red}$\times$ \\
      前方秘匿性 & $\rightarrow$ なし & \color{red}$\times$
    \end{tabular}
  \end{table}
\end{frame}

% --------------------------------------------------------------------
\begin{frame}[fragile]
  \frametitle{検証結果}
  \framesubtitle{前方秘匿性}

  \includegraphics[width=11cm]{graph/protocol1_forwardsecrecy.pdf}
\end{frame}

% ==============================================================================
\section{プロトコル\scriptsize$\beta$}

% --------------------------------------------------------------------
\begin{frame}
  \frametitle{対策・改善}
  \framesubtitle{オフライン攻撃と前方秘匿性}

  \begin{itemize}
    \item オフライン攻撃:\\~~~~ 弱い鍵から強い鍵を作る
    \vspace{1em}
    \item 前方秘匿性:\\~~~~ 通信毎に異なる共通鍵を使うようにする
  \end{itemize}
\end{frame}

% --------------------------------------------------------------------
\begin{frame}
  \frametitle{Diffie-Hellman鍵共有 (DH)}
  % \framesubtitle{DH鍵共有}

  \begin{enumerate}
    \item Aliceは乱数$a$を選択する
    \item Alice$\rightarrow$Bob : $A = g^a \pmod{p}$
    \item Bobは乱数$b$を選択する
    \item Bob$\rightarrow$Alice : $B = g^b \pmod{p}$
    \item AliceとBobは共通鍵$K$が求まる:\\~~~~$K = (g^a)^b = g^{ab} = (g^b)^a \pmod{p}$
  \end{enumerate}
  \vspace{2em}

  生成元$g$と素数$p$を適切に選ぶとき、盗聴者は公開値$A,B$から共有鍵$K$を求めることは困難(離散対数問題)
\end{frame}

% --------------------------------------------------------------------
\begin{frame}[fragile]
  \frametitle{プロトコルのモデル化}
  \framesubtitle{Diffie-Hellman鍵共有(DH鍵共有)}

  \begin{itemize}
    \item 一方向 :
      \\~~~~ {\color{gray}$A = g^{a} \pmod{p}$}
      \\~~~~ $A = \exp(g, a)$
    \item 公開値から共通鍵が求まる :
      \\~~~~ {\color{gray}$K = (g^a)^b = g^{ab} = (g^b)^a \pmod{p}$}
      \\~~~~ $K = \exp(\exp(g,a),b) = \exp(\exp(g,b),a)$
  \end{itemize}

  \vspace{-1em}
  \begin{center}
    \begin{tikzpicture}[color=white!30!blue]
      \draw[-{Triangle[width=35pt,length=15pt]}, line width=20pt](0,0) -- (0,-1)
        node [midway, right=1em] {ProVerifのコード};
    \end{tikzpicture}
  \end{center}
  \vspace{-2.5em}

  \begin{lstlisting}[basicstyle=\ttfamily\footnotesize]
    type G.
    type exponent.
    const g: G [data].  (* 生成元g *)
    fun exp(G, exponent): G.
    equation forall a: exponent, b: exponent;
      exp(exp(g,a),b) = exp(exp(g,b),a).
  \end{lstlisting}
\end{frame}

% --------------------------------------------------------------------
\begin{frame}
  \frametitle{プロトコル{$\beta$}}
  \framesubtitle{DH鍵共有$+$暗号化}

  \begin{center}
    \begin{tikzpicture}[node distance=2cm,auto,>=stealth']
      \node[draw=black] (server) {Server};
      \node[draw=black, left of=server, node distance=5cm] (client) {Client};
      \node[below of=server, node distance=6.5cm] (server_ground) {};
      \node[below of=client, node distance=6.5cm] (client_ground) {};
      %
      \draw (client) -- (client_ground);
      \draw (server) -- (server_ground);
      \node at ($(client)!0.10!(client_ground)$) [fill=white] {パスワード$p$, 平文$m$};
      \node at ($(server)!0.10!(server_ground)$) [fill=white] {パスワード$p$};
      \node at ($(client)!0.25!(client_ground)$) [fill=white] (Cg) {生成元$g = \mathrm{genG}(p)$};
      \node at ($(server)!0.25!(server_ground)$) [fill=white] (Sg) {生成元$g = \mathrm{genG}(p)$};
      \draw[->] ($(client)!0.35!(client_ground)$) -- node[above,scale=1,pos=0.3]{$g^a$} ($(server)!0.45!(server_ground)$);
      \draw[<-] ($(client)!0.45!(client_ground)$) -- node[above,scale=1,pos=0.7]{$g^b$} ($(server)!0.35!(server_ground)$);
      \node at ($(client)!0.52!(client_ground)$) [fill=white] (Cgab) {$g^{ab}$};
      \node at ($(server)!0.52!(server_ground)$) [fill=white] (Sgab) {$g^{ab}$};
      \node at ($(client)!0.70!(client_ground)$) [fill=white] (Css) {共有鍵$s = \mathrm{KDF}(g^{ba})$};
      \node at ($(server)!0.70!(server_ground)$) [fill=white] (Sss) {共有鍵$s = \mathrm{KDF}(g^{ab})$};
      \draw[->] ($(client)!0.85!(client_ground)$) -- node[above,scale=1,midway]{暗号化$\enc(m, s)$} ($(server)!0.90!(server_ground)$);
      \node at ($(server)!1.00!(server_ground)$) [fill=white] {復号};

      \onslide<2->

      % DH鍵共有と鍵導出KDFを枠で囲む
      \node[fit=(Cg)(Sg)(Cgab)(Sgab)(Css)(Sss), rounded corners=3pt, color=orange!30!white, line width=2pt, draw] {};
    \end{tikzpicture}
  \end{center}
\end{frame}

% --------------------------------------------------------------------
\begin{frame}[fragile]
  \frametitle{プロトコル{$\beta$}のモデル化}
  \framesubtitle{クライアント--サーバ間の通信}

  \begin{lstlisting}[basicstyle=\ttfamily\scriptsize]
let clientA() =
  new randomA: exponent;
  let gA = exp(genG(password), randomA) in
  out(c, gA); 
  in(c, gB: G);
  let sharedSecret = KDF(exp(gB, randomA)) in
  let ciphertext = enc(msg, sharedSecret) in
  out(c, ciphertext);
  0.

let serverB() =
  new randomB: exponent;
  let gB = exp(genG(password), randomB) in
  in(c, gA: G);
  out(c, gB);
  let sharedSecret = KDF(exp(gA, randomB)) in
  in(c, ciphertext: bitstring);
  let recvmsg = dec(ciphertext, sharedSecret) in
  0.

process
  ( (!clientA()) | (!serverB()) )
  \end{lstlisting}
\end{frame}

% --------------------------------------------------------------------
\begin{frame}[fragile]
  \frametitle{検証方法}
  \framesubtitle{秘匿性、オフライン攻撃、前方秘匿性}

  \begin{itemize}
    \item \code{attacker(v)}
      \\~~~~ 攻撃者は変数$v$に到達可能か
    \vspace{0.5em}
    \item \code{weaksecret v.}
      \\~~~~ 秘密の値$v$のエントロピーが低いとき、
      \\~~~~ 攻撃者は変数$v$に到達可能か
    \vspace{0.5em}
    \item \code{phase 1; out(c, password)}
      \vspace{0.5em}
      \\~~~~ Phase 0 : パスワードで平文$m$を暗号化して送信
      \\~~~~ Phase 1 : パスワードを漏洩させる
      \\~~~~~~ パスワード漏洩後に攻撃者は平文$m$に到達可能か
  \end{itemize}
\end{frame}

% --------------------------------------------------------------------
\begin{frame}[fragile]
  \frametitle{検証結果}
  \framesubtitle{秘匿性、オフライン攻撃}

  \begin{lstlisting}[frame=single]
$ proverif -color protocol2a.pv
...
--------------------------------------------------------------
Verification summary:

Query not attacker(msg[]) is %\color{OliveGreen}true%.

Query not attacker(password[]) is %\color{OliveGreen}true%.

Weak secret password is %\color{OliveGreen}true%.
  \end{lstlisting}

  \begin{table}
    \centering
    \begin{tabular}{lll}
      秘匿性 & $\rightarrow$ あり & \color{OliveGreen}$\checkmark$ \\
      オフライン攻撃 & $\rightarrow$ 困難 & \color{OliveGreen}$\checkmark$ \\
      \color{gray}前方秘匿性 &  &
    \end{tabular}
  \end{table}
\end{frame}

% --------------------------------------------------------------------
\begin{frame}[fragile]
  \frametitle{検証結果}
  \framesubtitle{前方秘匿性}

  \begin{lstlisting}[frame=single]
$ proverif -color protocol2b.pv
...
--------------------------------------------------------------
Verification summary:

Query not attacker_p1(msg[]) is %\color{OliveGreen}true%.

Query not attacker_p1(password[]) is %\color{red}false%.

...
  \end{lstlisting}

  \begin{table}
    \centering
    \begin{tabular}{lll}
      \color{gray}秘匿性 & \color{gray}$\rightarrow$ あり & \color{OliveGreen}$\checkmark$ \\
      \color{gray}オフライン攻撃 & \color{gray}$\rightarrow$ 困難 & \color{OliveGreen}$\checkmark$ \\
      前方秘匿性 & $\rightarrow$ あり & \color{OliveGreen}$\checkmark$
    \end{tabular}
  \end{table}
\end{frame}

% --------------------------------------------------------------------
\begin{frame}
  \frametitle{プロトコル$\beta$}
  \framesubtitle{Wi-Fiの新規格WPA3を参考に作成}

  \begin{center}
    \begin{tikzpicture}[node distance=2cm,auto,>=stealth']
      \node[draw=black] (server) {Server};
      \node[draw=black, left of=server, node distance=5cm] (client) {Client};
      \node[below of=server, node distance=6.5cm] (server_ground) {};
      \node[below of=client, node distance=6.5cm] (client_ground) {};
      %
      \draw (client) -- (client_ground);
      \draw (server) -- (server_ground);
      \node at ($(client)!0.10!(client_ground)$) [fill=white] {パスワード$p$, 平文$m$};
      \node at ($(server)!0.10!(server_ground)$) [fill=white] {パスワード$p$};
      \node at ($(client)!0.25!(client_ground)$) [fill=white] (Cg) {生成元$g = \mathrm{genG}(p)$};
      \node at ($(server)!0.25!(server_ground)$) [fill=white] (Sg) {生成元$g = \mathrm{genG}(p)$};
      \draw[->] ($(client)!0.35!(client_ground)$) -- node[above,scale=1,pos=0.3]{$g^a$} ($(server)!0.45!(server_ground)$);
      \draw[<-] ($(client)!0.45!(client_ground)$) -- node[above,scale=1,pos=0.7]{$g^b$} ($(server)!0.35!(server_ground)$);
      \node at ($(client)!0.52!(client_ground)$) [fill=white] (Cgab) {$g^{ab}$};
      \node at ($(server)!0.52!(server_ground)$) [fill=white] (Sgab) {$g^{ab}$};
      \node at ($(client)!0.70!(client_ground)$) [fill=white] (Css) {共有鍵$s = \mathrm{KDF}(g^{ba})$};
      \node at ($(server)!0.70!(server_ground)$) [fill=white] (Sss) {共有鍵$s = \mathrm{KDF}(g^{ab})$};
      \draw[->] ($(client)!0.85!(client_ground)$) -- node[above,scale=1,midway]{暗号化$\enc(m, s)$} ($(server)!0.90!(server_ground)$);
      \node at ($(server)!1.00!(server_ground)$) [fill=white] {復号};
    \end{tikzpicture}
  \end{center}
\end{frame}

% ==============================================================================
\section{おわりに}

% --------------------------------------------------------------------
\begin{frame}[fragile]
  \frametitle{おわりに}

  \begin{itemize}
    \item ProVerifは秘匿性や真正性を自動で検証可能
    \vspace{1em}
    \item いろんなプロトコルを検証してみると楽しい
  \end{itemize}

  \vspace{2em}
  \begin{center}
    Happy ProVerifying!
  \end{center}
\end{frame}


% ==============================================================================
% \appendix
\begin{frame}[allowframebreaks]
  \frametitle{参考文献}

  \begin{thebibliography}{9}
    \bibitem{proverif} Blanchet at el.: ProVerif 2.02pl1: Automatic Cryptographic Protocol Verifier, User Manual and Tutorial. INRIA, September 2020.
    \bibitem{proverif} Blanchet: ProVerif Automatic Cryptographic Protocol Verifier User Manual for Untyped Inputs. INRIA, September 2020.
  \end{thebibliography}

\end{frame}



% ==============================================================================

% % columns
% \begin{columns}
%   \begin{column}{.4\textwidth}
%     ...
%   \end{column}
%   \begin{column}{.6\textwidth}
%     ...
%   \end{column}
% \end{columns}
%
% % block, alertblock, exampleblock
% \begin{block}{見出し}
%   ...
% \end{block}
%
% % Overlay Specification
% \pause
% \begin{enumerate}
%   \item<1-> 1コマ目以降表示
%   \item<1-2> 1-2コマ目のみ表示
%   \item<2-3> 2-3コマ目のみ表示
% \end{enumerate}
% \uncover<1,3>{1コマ目と3コマ目のみ表示、スペースを占有する。}
% \only<2>{2コマ目のみ表示、スペースを占有しない。}
% \only<2|handout:0>{2コマ目のみ表示、handoutモードで非表示。}
%
% % Text animations (revealing one item per click)
% \begin{itemize}[<+->]
%   \item ...
% \end{itemize}

\end{document}
