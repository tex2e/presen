
\documentclass[dvipdfmx, 12pt]{beamer}
\usepackage{pgfpages}
\usepackage{bxdpx-beamer}
\usepackage{url}

%%%% PDFのしおりの文字化け対策 %%%%
\usepackage{atbegshi}
\ifnum 42146=\euc"A4A2
\AtBeginShipoutFirst{\special{pdf:tounicode EUC-UCS2}}
\else
\AtBeginShipoutFirst{\special{pdf:tounicode 90ms-RKSJ-UCS2}}
\fi
%%%%

%%%% スライドの見た目 %%%%
\usetheme{Singapore} % Theme
\usecolortheme{default} % Color
\usefonttheme{professionalfonts} % Font
%%%%

%%%% フォントの設定 %%%%
\usepackage[utf8]{inputenc} % UTF-8
\usepackage[T1]{fontenc} % 8bit font
\usepackage{lmodern} % Latin Modern font
\usepackage{otf} % otf (font)
\usepackage{bm} % bold math
\renewcommand{\kanjifamilydefault}{\gtdefault} % fontfamily: Gothic
\renewcommand{\familydefault}{\sfdefault} % 英語フォント
\usepackage{txfonts}
\usepackage{xcolor} % \textcolor{color}{text}
%%%%

% %% ----- math -----
% \usepackage{amsmath, amssymb}
% \renewcommand{\~}{$\sim$}
% %% ----- figure -----
% \usepackage{xcolor}
% \usepackage{here}
% \usepackage[dvipdfmx, hiresbb]{graphicx}
% %% ----- table -----
\usepackage{booktabs}
% \catcode63=\active \def?{\phantom{0}} % ? -> ' '
% \usepackage{tabularx} \newcolumntype{Y}{>{\centering\arraybackslash}X}
% \usepackage{multirow}
% %% ----- programing -----
\usepackage{verbatim}
% \newcommand{\code}[1]{ \texttt{\detokenize{#1}} }
% %% --- border ---
% \usepackage{fancybox, ascmac}
% %% ----- tikz -----
\usepackage{tikz}
\usetikzlibrary{fit, calc, positioning}

\title{N日でできる! TLS 1.3自作入門}
\subtitle{}
\author{@tex2e}
\institute{セキュリティ・キャンプ全国大会2019 LT大会}
\date{}

\begin{document}

% ==============================================================================
\begin{frame}
  \frametitle{}
  \titlepage %表紙
\end{frame}

% ==============================================================================
\begin{frame}
  \frametitle{今日のお話}
  \begin{center}
    \Huge TLS 1.3
  \end{center}
\end{frame}

\begin{frame}
  \frametitle{TLSとは}
  \begin{center}
    \large
    通信する2人はこれまでに\alert{会ったことがなく}、\\[3mm]
    \alert{安全ではない}通信路を使ったとしても、\\[3mm]
    \alert{安全に}やりとりができる
  \end{center}
\end{frame}

% ==============================================================================
\section{予備知識} % Preliminaries

\begin{frame}
  \frametitle{安全な通信路とは...}

  \begin{itemize}
    \item 真正性
      \begin{itemize}
        \item 通信相手が本物であることを確認できる
        \item (サーバ証明書による\alert{認証} ... X.509 Cert, PKI)
      \end{itemize}

    \vspace{0.5em}
    \item 機密性
      \begin{itemize}
        \item 権限を持つ人だけがアクセスできる
        \item (通信内容の\alert{暗号化} ... AES, ChaCha20)
      \end{itemize}

    \vspace{0.5em}
    \item 完全性
      \begin{itemize}
        \item 改ざんされない
        \item (認証付き暗号による\alert{改ざん検知} ... AEAD)
      \end{itemize}
  \end{itemize}
\end{frame}

\begin{frame}
  \frametitle{TLSのハンドシェイク}

  \begin{columns}
    \begin{column}{.7\textwidth}
      \begin{itemize}
        \item Handshake
        \begin{itemize}
          \item どの暗号スイートを使うか決める
          \item \alert{公開鍵暗号}を用いて鍵共有する
          \item 証明書を使って認証する
        \end{itemize}
        \vspace{0.5cm}
        \item Application Data
        \begin{itemize}
          \item \alert{共通鍵暗号}を用いて暗号化する
          \item HTTPを暗号化したデータなど
        \end{itemize}
      \end{itemize}
    \end{column}
    \begin{column}{.3\textwidth}
      \begin{figure}
        \footnotesize
        \centering
        \input{img/tls-handshake}
      \end{figure}
    \end{column}
  \end{columns}
\end{frame}

\begin{frame}
  \frametitle{TLS 1.3のやりとり}

  \begin{figure}
    \centering
    \input{img/tls-flow}
  \end{figure}
\end{frame}

% ==============================================================================
\section{実装} % Implementation
\begin{frame}
  \frametitle{プログラマの3大「嗜み」}

  \begin{itemize}
    \item 自作OS
    \vspace{5mm}
    \item 自作コンパイラ (プログラミング言語)
    \vspace{5mm}
    \item 自作プロトコルスタック (TCP/IP, TLS)
  \end{itemize}
  \vspace{5mm}

  \begin{center}
    \alert{※諸説あり}
  \end{center}
\end{frame}

\begin{frame}
  \frametitle{TLSどうやって作るの?}
  \begin{center}
    \Huge
    RFCを読む
  \end{center}
\end{frame}

\begin{frame}
  \frametitle{TLS 1.3 (RFC 8446)}
  \begin{figure}
    \centering
    \includegraphics[scale=0.25]{img/rfc8446}
  \end{figure}
\end{frame}

\begin{frame}[fragile]
  \frametitle{構造体とバイト列の相互変換}

  \footnotesize
  \begin{figure}
    \centering
    \begin{tikzpicture}[> = latex, auto]
      \node (fig) {\includegraphics[scale=0.35]{img/tls13-wireshark.png}};
      \draw[thick,->] (fig.10)  to[bend left] node {変換} (fig.340);
      \draw[thick,->] (fig.200) to[bend left] node {復元} (fig.170);
    \end{tikzpicture}
  \end{figure}
\end{frame}

\begin{frame}
  \frametitle{実装の流れ}

  TLSのやりとりの実装:

  \begin{enumerate}
    \item ソケット通信
    \item メッセージの構造体とバイト列の相互変換
  \end{enumerate}
  \vspace{5mm}

  TLSのやりとりの中身の実装:

  \begin{enumerate}
    \item 楕円曲線Diffie-Hellman鍵共有
    \item HKDFによる鍵スケジューリング
    \item 認証付き暗号(AEAD)
    \item X.509証明書
  \end{enumerate}
\end{frame}

% ==============================================================================
\section{最新動向}
\begin{frame}
  \frametitle{TLS 1.3の最新動向 (Server/Client)}

  \begin{figure}
    \centering
    \includegraphics[scale=0.18]{img/ssl-pulse.png}
  \end{figure}

  \begin{figure}
    \centering
    \includegraphics[scale=1.15]{img/can-i-use-tls13.png}
  \end{figure}
\end{frame}

\begin{frame}
  \frametitle{TLS 1.3の最新動向 (QUIC)}

  \begin{center}
    UDPでコネクション確立とTLS 1.3確立を同時に行う
  \end{center}

  \begin{columns}
    \begin{column}{.5\textwidth}
      \begin{figure}
        \centering
        \scalebox{.9}{\input{img/http-over-tcp-tls.tex}}
      \end{figure}
    \end{column}
    \begin{column}{.5\textwidth}
      \begin{figure}
        \centering
        \scalebox{.9}{\input{img/http-over-quic.tex}}
      \end{figure}
    \end{column}
  \end{columns}
\end{frame}

% ==============================================================================
\section{まとめ} % Conclusion
\begin{frame}
  \frametitle{TLS 1.3自作は楽しいけど難しい}
  \begin{itemize}
    \item 文書はほとんど\alert{英語}
    \item 暗号技術の基盤となる\alert{数学}の知識
    \item \alert{ネットワーク}技術の知識
    \item \alert{RFC}は入門書ではないので初学者には厳しい
  \end{itemize}
\end{frame}

\begin{frame}
  \begin{center}
    \Huge
    30日で TLS 1.3 は\\作れないよ
  \end{center}
\end{frame}

\begin{frame}
  \begin{center}
    \Huge
    おわり
  \end{center}
\end{frame}

% ==============================================================================
\appendix
\begin{frame}[allowframebreaks]
  \frametitle{参考文献}

  \begin{thebibliography}{9}
    \bibitem{tls13} RFC 8446: The Transport Layer Security (TLS) Protocol Version 1.3. IETF, August 2018.
    \bibitem{owasp} Andy Brodie: Overview of TLS v1.3. OWASP, 2017. URL \url{https://www.owasp.org/images/d/d3/TLS_v1.3_Overview_OWASP_Final.pdf}
    \bibitem{ssllabs} SSL Labs: SSL Pulse. Qualys, Inc, June 2019. URL \url{https://www.ssllabs.com/ssl-pulse/}
    \bibitem{caniuse} @Fyrd, @Lensco: Can I use... URL \url{https://caniuse.com/}
    \bibitem{quic} IETF Draft: ``QUIC: A UDP-Based Multiplexed and Secure Transport''. URL \url{https://tools.ietf.org/html/draft-ietf-quic-transport-22}
    \bibitem{cloudflare} Alessandro Ghedini: The Road to QUIC. Cloudflare, Inc, 2018. URL \url{https://blog.cloudflare.com/the-road-to-quic/}
    \bibitem{pandax381} 雅也 山本: TCP/IPプロトコルスタック自作入門. KLab Inc, 2018. URL \url{https://www.slideshare.net/pandax381/tcpip-105857327}
    \bibitem{lambdanote} Ivan Ristić 著, 齋藤孝道 監訳: プロフェッショナルSSL/TLS. ラムダノート, 2018.
  \end{thebibliography}

\end{frame}

% % this frame won't be included in the handout mode
% \begin{frame}<handout:0>
% I am a lolcat!
% \end{frame}

% % columns
% \begin{columns}
%   \begin{column}{.4\textwidth}
%     ...
%   \end{column}
%   \begin{column}{.6\textwidth}
%     ...
%   \end{column}
% \end{columns}
%
% % block, alertblock, exampleblock
% \begin{block}{見出し}
%   ...
% \end{block}
%
% % Overlay Specification
% \pause
% \begin{enumerate}
%   \item<1-> 1コマ目以降表示
%   \item<1-2> 1-2コマ目のみ表示
%   \item<2-3> 2-3コマ目のみ表示
% \end{enumerate}
% \uncover<1,3>{1コマ目と3コマ目のみ表示、スペースを占有する。}
% \only<2>{2コマ目のみ表示、スペースを占有しない。}
% \only<2|handout:0>{2コマ目のみ表示、handoutモードで非表示。}
%
% % Text animations (revealing one item per click)
% \begin{itemize}[<+->]
%   \item ...
% \end{itemize}

\end{document}
